\documentclass[12pt,a4paper]{article}

% ===============================================
% PAQUETES ESENCIALES
% ===============================================
\usepackage[utf8]{inputenc}        % Codificación UTF-8
\usepackage[spanish]{babel}        % Idioma español
\usepackage[T1]{fontenc}           % Codificación de fuentes
\usepackage{lmodern}               % Fuente moderna
\usepackage{microtype}             % Mejora tipografía

% ===============================================
% CONFIGURACIÓN DE PÁGINA
% ===============================================
\usepackage[left=2.5cm, right=2.5cm, top=3cm, bottom=3cm]{geometry}
\usepackage{fancyhdr}              % Encabezados y pies de página
\usepackage{setspace}              % Interlineado

% ===============================================
% PAQUETES MATEMÁTICOS
% ===============================================
\usepackage{amsmath}               % Matemáticas avanzadas
\usepackage{amsfonts}              % Fuentes matemáticas
\usepackage{amssymb}               % Símbolos matemáticos
\usepackage{mathtools}             % Herramientas matemáticas

% ===============================================
% PAQUETES GRÁFICOS
% ===============================================
\usepackage{graphicx}              % Insertar imágenes
\usepackage{float}                 % Control de posición de figuras
\usepackage{subfig}                % Subfiguras
\usepackage{xcolor}                % Colores
\usepackage{tikz}                  % Gráficos vectoriales

% ===============================================
% PAQUETES DE TABLAS Y LISTAS
% ===============================================
\usepackage{booktabs}              % Tablas profesionales
\usepackage{tabularx}              % Tablas con ancho fijo
\usepackage{longtable}             % Tablas largas
\usepackage{enumerate}             % Enumeraciones personalizadas
\usepackage{enumitem}              % Control de listas

% ===============================================
% PAQUETES DE CÓDIGO
% ===============================================
\usepackage{listings}              % Insertar código
\usepackage{verbatim}              % Texto literal

% ===============================================
% OTROS PAQUETES ÚTILES
% ===============================================
\usepackage{url}                   % URLs
\usepackage{hyperref}              % Enlaces internos y externos
\usepackage{cite}                  % Citas bibliográficas
\usepackage{appendix}              % Apéndices

% ===============================================
% CONFIGURACIONES PERSONALIZADAS
% ===============================================

% Configurar encabezados
\pagestyle{fancy}
\fancyhf{}
\fancyhead[L]{\leftmark}
\fancyhead[R]{\thepage}
\renewcommand{\headrulewidth}{0.4pt}

% Configurar código
\lstset{
    language=Python,
    basicstyle=\ttfamily\small,
    keywordstyle=\color{blue},
    commentstyle=\color{green!50!black},
    stringstyle=\color{red},
    numbers=left,
    numberstyle=\tiny,
    frame=single,
    breaklines=true,
    breakatwhitespace=true
}

% Configurar enlaces
\hypersetup{
    colorlinks=true,
    linkcolor=blue,
    filecolor=magenta,
    urlcolor=cyan,
    citecolor=green
}

% ===============================================
% INFORMACIÓN DEL DOCUMENTO
% ===============================================
\title{Título de tu Documento}
\author{Tu Nombre}
\date{\today}

% ===============================================
% DOCUMENTO
% ===============================================
\begin{document}

% Portada
\maketitle
\thispagestyle{empty}

% Índice
\newpage
\tableofcontents
\newpage

% ===============================================
% CONTENIDO PRINCIPAL
% ===============================================

\section{Introducción}

Este es un ejemplo de plantilla LaTeX completa con los paquetes y configuraciones más comunes.

\subsection{Subsección de ejemplo}

Aquí puedes escribir el contenido de tu documento. Puedes usar \textbf{texto en negrita}, \textit{texto en cursiva}, \underline{texto subrayado}.

\section{Matemáticas}

\subsection{Ecuaciones inline}
Puedes escribir ecuaciones en línea como $E = mc^2$ o $\sum_{i=1}^{n} x_i$.

\subsection{Ecuaciones en display}
También puedes usar ecuaciones en display:

\begin{equation}
    \int_{-\infty}^{\infty} e^{-x^2} dx = \sqrt{\pi}
\end{equation}

\begin{align}
    f(x) &= ax^2 + bx + c \\
    f'(x) &= 2ax + b
\end{align}

\section{Figuras y Tablas}

\subsection{Figuras}

\begin{figure}[H]
    \centering
    % \includegraphics[width=0.5\textwidth]{imagen.png}
    \caption{Ejemplo de figura}
    \label{fig:ejemplo}
\end{figure}

\subsection{Tablas}

\begin{table}[H]
    \centering
    \caption{Ejemplo de tabla}
    \label{tab:ejemplo}
    \begin{tabular}{lcr}
        \toprule
        Columna 1 & Columna 2 & Columna 3 \\
        \midrule
        Dato 1    & Dato 2    & Dato 3    \\
        Dato 4    & Dato 5    & Dato 6    \\
        \bottomrule
    \end{tabular}
\end{table}

\section{Listas}

\subsection{Lista con viñetas}
\begin{itemize}
    \item Primer elemento
    \item Segundo elemento
    \item Tercer elemento
\end{itemize}

\subsection{Lista numerada}
\begin{enumerate}
    \item Primer elemento
    \item Segundo elemento
    \item Tercer elemento
\end{enumerate}

\section{Código}

\subsection{Código inline}
Puedes usar código inline como \verb|print("Hola mundo")|.

\subsection{Bloque de código}
\begin{lstlisting}[caption=Ejemplo de código Python]
def saludo(nombre):
    """
    Función que saluda a una persona
    """
    print(f"Hola {nombre}!")
    return f"Saludos para {nombre}"

# Llamar a la función
resultado = saludo("Mundo")
print(resultado)
\end{lstlisting}

\section{Referencias}

Puedes referenciar figuras como la Figura~\ref{fig:ejemplo} o tablas como la Tabla~\ref{tab:ejemplo}.

También puedes crear referencias a secciones. Por ejemplo, ve la Sección~\ref{sec:conclusion}.

\section{Conclusión}
\label{sec:conclusion}

Esta plantilla incluye los elementos más comunes que necesitarás en un documento LaTeX. Puedes modificar y adaptar según tus necesidades específicas.

% ===============================================
% BIBLIOGRAFÍA (opcional)
% ===============================================
\begin{thebibliography}{9}
\bibitem{ejemplo1}
Apellido, N. (2023). \textit{Título del libro}. Editorial.

\bibitem{ejemplo2}
Autor, A. (2022). Título del artículo. \textit{Revista Científica}, 15(3), 123-145.
\end{thebibliography}

% ===============================================
% APÉNDICES (opcional)
% ===============================================
\appendix
\section{Apéndice A: Información adicional}

Aquí puedes incluir información adicional, código extra, tablas extensas, etc.

\end{document}